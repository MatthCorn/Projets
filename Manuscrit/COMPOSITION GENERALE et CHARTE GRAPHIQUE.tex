\chapter{COMPOSITION GÉNÉRALE, CHARTE GRAPHIQUE}
\section{COMPOSITION DU DOCUMENT}
Les deux premières pages sont consacrées aux informations institutionnelles. \\ \par
Une troisième page peut être ajoutée pour compléter les informations institutionnelles réglementaires des deux premières pages. Par exemple, pour donner des informations sur l’organisme d’accueil ou financeur et afficher leurs logos, pour décrire brièvement un cadre partenarial, pour fournir les noms de personnalités invitées à siéger aux cotés du Jury pour la soutenance, pour afficher le logo du laboratoire etc. \\ \par 
La page des remerciements est alors placée en 3\ieme ou 4\ieme page, selon qu’une 3\ieme page a été ajoutée ou non pour apporter ces compléments d’informations.
\section{QUELS LOGOS FAIRE FIGURER ?}
Il ne doit figurer sur la \textbf{page de couverture de thèse}, aucun autre logo que le \textbf{logo de l’université Paris-Saclay} et, en cas de cotutelle internationale de thèse, le logo de l’université partenaire étrangère qui délivre également le diplôme de doctorat pour cette thèse. \\ \par
Il ne doit figurer sur la \textbf{seconde page}, aucun autre logo que le \textbf{logo de l’école doctorale}.
Les logos institutionnels en vigueur de l’université Paris-Saclay et des écoles doctorales sont fournis au paragraphe 7.2.\\ \par
Les autres logos, comme celui du laboratoire, d’une entreprise, d’une composante, d’un établissement-composante, d’une université membre associée, d’un organisme de recherche ou de toute autre organisation partenaire de la thèse, peuvent être regroupés dans une troisième page intérieure, avant la page des remerciements, mais ne doivent pas figurer pas sur les deux premières pages.
\section{POLICES DE CARACTÈRES ET COULEURS}
Les polices de caractère à utiliser sont : Open Sans ou Segoe UI ou Tahoma ou Ebrima. Il ne faut utiliser qu’\textbf{une seule police de caractère}.\\ \par
Sur les 3 premières pages, seules deux couleurs de police sont utilisées, noir et prune (R : 99 V : 0 B : 60). Dans le reste du document, vous pouvez utiliser d’autres couleurs de police, si nécessaire, en veillant à ce qu’elles appartiennent à la palette de couleurs de la charte graphique de l’université Paris-Saclay.  D’autres nuances de couleurs peuvent être utilisées parmi les nuances de la palette de l’UPSaclay.\\ \par

\begin{figure}
\begin{center} 
\includegraphics[scale=0.6]{Charte_graphique_ups.png}
\end{center}
\caption{Palette de couleurs de la charte graphique}
\end{figure}
La \href{https://portail.universite-paris-saclay.fr/communication/Pages/Charte-graphique.aspx}{charte graphique de l’Université} peut être téléchargée sur l’intranet pour plus d’information.\\ \par
Sur la couverture de la thèse, le \textbf{titre} est en police normale de taille 20, de couleur prune et la \textbf{traduction du titre} est en police normale de taille 12, de couleur noire et en italique. Si le titre et sa traduction sont très longs, la police peut éventuellement être réduite, mais sans descendre en dessous d’une police 14 pour le titre et d’une police 10 pour la traduction du titre.
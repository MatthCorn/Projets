\chapter{BIEN CITER SES SOURCES}
La citation des sources fait partie intégrante du travail scientifique et participe de sa qualité et de son intégrité.
\section{S'INFORMER SUR LE PLAGIAT}
Des mauvaises pratiques de citation peuvent conduire, même sans le vouloir, au plagiat. Le copier/coller, la paraphrase, la réutilisation d’images ou d’idées sans citer la source sont des situations de plagiat (n’hésitez pas à regarder cette courte vidéo sur les différentes formes de plagiat, volontaires ou non : \url{https://infotrack.unige.ch/comment-reconnaitre-les-cas-de-plagiat)}\\ \par
Chaque discipline possède ses propres normes en termes de citation des sources. Renseignez-vous auprès de vos pairs pour connaître le style bibliographique et le style de citation à privilégier. Nous vous encourageons vivement à utiliser un logiciel de gestion bibliographique tel que Zotero. Vous pouvez retrouver des supports de formation à ce logiciel dans l’espace eCampus Doctorat, ouvert à tou·te·s sur auto-inscription : \url{https://ecampus.paris-saclay.fr/course/view.php?id=36678}

\section{LES IMAGES}
Vous pouvez réutiliser dans votre thèse des images provenant d’articles ou de livres protégés par un copyright. Cela relève en effet de l’exception pédagogique, une des exceptions au droit d’auteur. Attention cependant, vous ne pouvez pas faire ce que vous voulez de cette image ! La loi vous autorise à inclure jusqu’à 20 images (en 720 dpi) sans demander d’autorisation à l’auteur. En revanche, une autorisation est nécessaire à partir de la 21\ieme image. Les sources des images doivent être mentionnées et aucune modification n’est autorisée.\\ \par
Pour une présentation détaillée des différents cas d’utilisation d’images dans les thèses et les travaux universitaires, voir : \url{https://ethiquedroit.hypotheses.org/2947}
\section{ARTICLES JOINTS A LA THÈSE}
Vous pouvez joindre vos articles à votre thèse. Toutefois, si votre thèse est diffusée en ligne (tout de suite après la soutenance, après un embargo ou après une période de confidentialité), il convient de s’assurer que vous respectez bien les politiques des éditeurs. En effet, tous n’autorisent pas la diffusion en accès libre de la version éditeur des articles. Utilisez \href{https://v2.sherpa.ac.uk/romeo/}{Sherpa Romeo} pour connaître la politique des éditeurs. \\ \par
Selon la \href{https://www.legifrance.gouv.fr/dossierlegislatif/JORFDOLE000031589829/}{Loi pour une république numérique}, si votre recherche est financée à au moins 50\% par des fonds publics français, vous avez le droit, en tant qu’auteur, de diffuser la version acceptée de l’article (mais sans la mise en pages de l’éditeur) au bout de 6 mois après publication pour les articles en sciences, techniques et médecine et 12 mois pour les articles en sciences humaines et sociales. Et ce quelque soit la politique éditoriale de l’éditeur.\\ \par
Pour plus d’informations sur cette question, consultez la section « Déposer dans une archive ouverte » du \href{https://www.ouvrirlascience.fr/wp-content/uploads/2021/09/Passeport-pour-la-science-ouverte-Guide-a-lusage-des-doctorants_10-09-2021_WEB_FR.pdf}{Passeport pour la science ouverte}.

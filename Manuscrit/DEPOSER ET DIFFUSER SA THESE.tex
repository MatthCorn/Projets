\chapter{DÉPOSER ET DIFFUSER SA THÈSE}
La thèse fait l’objet d’un dépôt légal, en deux étapes, avant la remise du manuscrit aux rapporteurs et après la soutenance, qui protège le droit d’auteur du docteur.\\ \par 
Elle fait ensuite l’objet d’une diffusion sur le portail national des thèses \url{www.theses.fr} et le portail européen des thèses \href{https://www.dart-europe.org/basic-search.php}{DART-Europe}, sauf si la thèse présente un caractère confidentiel avéré.
\section{LES RESSOURCES A CONSULTER}

\begin{itemize}
\renewcommand{\labelitemi}{$\bullet$}
\item \href{http://corist-shs.cnrs.fr/sites/default/files/ressources/droit_auteur_lecture_vf.pdf}{Je publie, Quels sont mes droits ?}
\item Le cadre réglementaire du \href{https://www.legifrance.gouv.fr/codes/article_lc/LEGIARTI000006845515/}{dépôt légal} sur Légifrance
\end{itemize}

\section{LES DÉMARCHES}
Retrouvez le détail des démarches du dépôt de thèse dans cette \href{https://www.universite-paris-saclay.fr/sites/default/files/2021-12/fiche-depot-legal-these-2021_0.pdf}{fiche explicative}\\ \par
Si la thèse présente un caractère confidentiel avéré, le classement confidentiel de la thèse et, si nécessaire, une dérogation au caractère public de la soutenance (huis-clos) peuvent être \href{https://www.universite-paris-saclay.fr/research/doctorate/quality-assurance-documents/documents-de-reference-relatifs-la-soutenance-de-la-these}{demandés au chef d’établissement}.

\newgeometry{top=4cm, bottom=4cm, left=4cm, right=4cm}

\chapter{CIVILITÉ, FÉMINISATION DES TITRES ET FONCTIONS}
Il est recommandé de ne pas indiquer les civilités (Madame / Monsieur) ni pour le docteur ou la docteure, ni pour les membres du Jury ou de l’équipe d’encadrement.\\ \par 
Toutefois, si cette recommandation n’était pas suivie, il faudrait alors assurer l’homogénéité. La civilité devrait alors être précisée pour \textbf{toutes les personnes} qui figurent sur la couverture (docteur.e, membres du Jury ou de l’équipe d’encadrement) en utilisant les \textbf{mêmes conventions} pour tous (Madame / Monsieur ou Mme / M.)\\ \par
Il est recommandé de féminiser les titres des membres du Jury ou de l’équipe d’encadrement (Professeur / Professeure, Maître ou Maîtresse de conférences etc.) ainsi que les fonctions tenues dans le Jury (examinateur / examinatrice ou Présidente / Présidente). \\ \par
Pour « rapporteur », la forme féminine n’est pas recommandée faute de stabilisation. Si la personne concernée souhaitait la forme féminine, il faudrait alors lui demander de préciser la forme (« rapporteure » ou « rapporteuse » ?) qu’elle préfère voir figurer sur la couverture.
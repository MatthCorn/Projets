\chapter{INFORMATIONS GÉNÉRALES SUR LA PAGE DE COUVERTURE}
Les informations figurant sur la page de couverture de la thèse doivent être cohérentes avec le diplôme et avec les métadonnées de la thèse sur le portail nationale des thèses \url{www.theses.fr}.
\section{TITRE DE LA THÈSE ET LANGUE(S)}
Le \textbf{titre de la thèse} doit être fourni en \textbf{français} et en \textbf{anglais}. Par défaut, le titre est en français et la traduction du titre est en anglais. Cependant, lorsque la thèse est rédigée en anglais, le titre peut être fourni en anglais et la traduction en français.\\ \par
Les affiliations (université de rattachement…) peuvent, le cas échéant, être fournies en anglais pour des membres étrangers du Jury. La langue par défaut restant le français.\\ \par
Tous les autres éléments de la couverture de la thèse sont en français, les noms des entités (école doctorale, unité de recherche, référent etc.) ainsi que les titres des membres du jury (Professeur, Maître de Conférences etc.). Les correspondances entre titres étrangers et français peuvent être trouvées sur le site du ministère (\href{https://www.galaxie.enseignementsup-recherche.gouv.fr/ensup/pdf/EC_pays_etrangers/Tableau_comparaison_au_26_septembre_2012.pdf}{GALAXIE})\footnote{\url{https://www.galaxie.enseignementsup-recherche.gouv.fr/ensup/pdf/EC_pays_etrangers/Tableau_comparaison_au_26_septembre_2012.pdf}}.
\section{SPÉCIALITÉ DE DOCTORAT}
La spécialité de doctorat doit faire partie des spécialités pour lesquelles l’école doctorale est accréditée (en pratique : cela implique que vous devez pouvoir la sélectionner dans le menu déroulant des spécialités dans Adum). \\ \par
La spécialité de doctorat retenue, via le menu déroulant dans Adum, sera celle qui figurera sur le diplôme.\\ \par
Si votre spécialité n’apparaît pas, il faut contacter le directeur de votre école doctorale.
\newpage
\section{UNITÉ DE RECHERCHE}
L’unité de recherche dans laquelle la thèse a été préparée est précisée sur la couverture de la thèse. Le nom de l’unité est cité en respectant les règles de signature officielles, telles qu’elles ont été convenues entre les tutelles des unités de recherche liées à l’université Paris-Saclay.\\ \par
\textbf{Pour les trouver} : il faut sélectionner votre unité de recherche via la barre de sélection depuis cette page web : \url{https://www.universite-paris-saclay.fr/fr/signature} et copier-coller l’adresse de l’unité de recherche sur la couverture de thèse. 
Puis mettre l’acronyme officiel en  premier et le nom des tutelles ensuite, entre parenthèses, dans l’ordre où elles sont indiquées sur \url{https://www.universite-paris-saclay.fr/fr/signature}.
Par exemple, pour IJCLab :

\begin{itemize}
\renewcommand{\labelitemi}{$\bullet$}
\item Voici ce qu’on récupère par un copié-collé depuis l’adresse ci-dessus : « \textit{Université Paris-Saclay, CNRS, IJCLab, 91405, Orsay, France} ».
\item Voici comment faire la citation sur la couverture de thèse : « \textit{IJCLab (Université Paris-Saclay, CNRS)}».
\end{itemize}

Si la thèse a été préparée dans deux unités de recherche (travaux interdisciplinaires, cotutelle internationale, mobilité…) merci de citer les deux unités de recherche.\\ \par
Si vous êtes doctorant de l'université Paris-Saclay mais ne trouvez pas votre unité dans la liste, votre unité ne fait probablement pas partie de l'université. Dans ce cas, et à défaut d'une recommandation commune entre l'université et votre unité, complétez la ligne "unité de recherche" de votre page de titre en suivant les recommandations de votre unité de recherche.\\ \par
La mention de l'université Paris-Saclay comme établissement de soutenance de votre thèse sera automatique en utilisant le modèle de page de couverture de l'université Paris-Saclay.

\section{LE RÉFÉRENT}
Les référents sont à choisir, en cohérence avec ce qui figure dans votre dossier d’inscription, parmi les composantes, établissements-composantes et universités membres associés de l’Université Paris-Saclay :

\begin{itemize}
%\renewcommand{\labelitemi}{$\bullet$}
\item Faculté de droit, économie et gestion, 
\item Faculté de médecine
\item Faculté de pharmacie
\item Faculté des sciences d’Orsay
\item Faculté des sciences du sport 
\item AgroParisTech
\item Institut d’Optique
\item ENS Paris-Saclay
\item CentraleSupélec
\item Université de Versailles-Saint-Quentin-en-Yvelines
\item Université d'Évry Val d’Essonne
\item École Nationale d’Architecture de Versailles 
\end{itemize}

\section{GRADUATE SCHOOL}
La Graduate School est à choisir en cohérence avec votre sujet de thèse et ce qui figure dans votre dossier d’inscription, parmi la ou les Graduate Schools de l’Université Paris-Saclay de rattachement de votre école doctorale ou de votre pôle d’école doctorale :

\begin{itemize}
\renewcommand{\labelitemi}{$\bullet$}
\item Biosphera
\item Chimie
\item Informatique et sciences du numérique
\item Droit
\item Économie - Management
\item Géosciences, climat, environnement et planètes
\item Humanités et Sciences du Patrimoine
\item Life Sciences and Health
\item Mathématiques
\item Physique
\item Santé et médicaments
\item Santé publique
\item Sciences de l’ingénierie et des systèmes
\item Sociologie et Science Politique
\item Sport, mouvement et facteurs humains
\end{itemize}

\section{ÉCOLE DOCTORALE}

\begin{itemize}
\renewcommand{\labelitemi}{$\bullet$}
\item n°127 : astronomie et astrophysique d'Île-de-France (AAIF)
\item n°129 : sciences de l'environnement d’Île-de-France (SEIF)
\item n°564 : physique en Île-de-France (PIF)
\item n°566 : sciences du sport, de la motricité et du mouvement humain (SSMMH)
\item n°567 : sciences du végétal : du gène à l'écosystème (SEVE)
\item n°568 : signalisations et réseaux intégratifs en biologie (Biosigne)
\item n°569 : innovation thérapeutique : du fondamental à l'appliqué (ITFA)
\item n°570 : santé publique (EDSP)
\item n°571 : sciences chimiques : molécules, matériaux, instrumentation et biosystèmes (2MIB)
\item n°572 : ondes et matière (EDOM)
\item n°573 : interfaces : matériaux, systèmes, usages (INTERFACES)
\item n°574 : mathématiques Hadamard (EDMH)
\item n°575 : electrical, optical, bio : physics and engineering  (EOBE)
\item n°576 : particules hadrons énergie et noyau : instrumentation, imagerie, cosmos et simulation (PHENIICS)
\item n°577 : structure et dynamique des systèmes vivants (SDSV)
\item n°579 : sciences mécaniques et énergétiques, matériaux et géosciences  (SMEMaG)
\item n°580 : sciences et technologies de l'information et de la communication (STIC)
\item n°581 : agriculture, alimentation, biologie, environnement, santé (ABIES)
\item n°582 : cancérologie : biologie - médecine - santé (CBMS)
\item n°629 : Sciences sociales et humanités (SSH)
\item n°630 : Droit,Économie,Management (DEM)
\end{itemize}

\section{LIEU ET DATE DE SOUTENANCE}
Au moment de l’annonce de soutenance, le lieu et la date de soutenance, servent à donner au public toutes les informations nécessaires pour assister à la soutenance. Étant donné que les soutenances de doctorat doivent être publiques. Il faut donc une information détaillée permettant au public d’y accéder, précisant ainsi l’horaire de début de la soutenance, la salle, l’adresse physique en présentiel ou le lien d’accès à la salle virtuelle lorsque la soutenance se tient en visioconférence ou les deux.\\ \par
En revanche, \textbf{pour la couverture de la thèse} et le dépôt légal de la thèse, le lieu et la date de soutenance ont une fonction « légale » : le lieu définit de quelle juridiction relève le dépôt légal de la thèse et la date est utile, par exemple pour définir l’antériorité ou la fin d’une période de confidentialité. L’information doit donc être donnée sous une forme beaucoup plus synthétique que dans l’annonce de soutenance.\\ \par
Sur la couverture de la thèse, la date doit être fournie au format « \textbf{JJ Mois AAA} » et le lieu de soutenance est simplement la ville, la commune ou la communauté d’agglomérations où s’est tenue la soutenance. Lorsque la soutenance a eu lieu dans les locaux de l’université Paris-Saclay, le lieu à indiquer est celui de la communauté d’agglomérations où se trouve le siège de l’université Paris-Saclay, à savoir « \textbf{Paris-Saclay} », que la thèse ait eu lieu en présentiel ou en visioconférence.\\ \par
Exemple : Thèse soutenue à Paris-Saclay,  le 10 Mars 2021.
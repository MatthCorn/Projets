\chapter{Problématique (plan)}

Le chapitre sur la problématique contiendra les éléments suivants:
\begin{itemize}
\item Description du fonctionnement du capteur ESM dans l'environnement (dans la limite de ce qu'on peut dire), intégré dans la chaine algorithmique de traitement de l'information.
\item Description du fonctionnement de l'environnement numérique, avec l'explication des modélisations de chaque traitement.
\item Spécification du goulot d'étranglement et commentaire sur les données I/O.
\item Commentaire sur le complexité du problème pour l'apprentissage automatique : double problématique génération et traitement de séquence.
\item \textcolor{red}{Penser à ajouter des références de traitement avec ce principe de mesureur (brevet ?) pour montrer qu'on ne révèle pas des secrets}
\end{itemize}

\chapter{Problématique}
\section{Introduction}
Comme abordé dans le chapitre introductif, notre mission est d'accélérer la simulation d'un environnement numérique modélisant l'interception d'impulsion RADAR par des capteurs ESM. Ce chapitre va nous permettre de revenir sur cette problématique. Nous commencerons par l'explication du fonctionnement du capteur ESM et son intégration dans le chaine de traitement de l'information. Nous verrons ensuite pourquoi il est nécessaire de disposer d'un environnement numérique modélisant le fonctionnement du capteur et nous reviendrons sur cette modélisation. APrès nous identifierons le goulot d'étranglement et précierons alors la nature exacte du problème d'accélération que nous aurons à résoudre. Nous porterons une attention particulière à la lecture de se problème sous le spectre de l'apprentissage automatique en notant que l'aspect est à la fois traitement de séquence et génération. Finalement, nous exposerons les références des problèmes où ce type de traitement apparaît, des brevets existants, etc

\section{Description de l'intégration du capteur ESM et son fonctionnement}

\subsection{Contexte d'utilisation}
guerre électronique, objectif : intercepté les émissions RADAR pour détecter les émetteurs et les identifier. objectif long terme: décision stratégique, maintient en vie de l'appareil 

Dans un scénario de guerre, un avion va recevoir les impulsions des radars présents dans son environnement, eux-mêmes cherchant à identifier leurs environnements. Les caractéristiques de ces impulsions sont une grande source d'informations et permettent d'identifier le radar de provenance. Ainsi, munis d'antennes et capteurs servant à une écoute passive des impulsions radars (capteurs de guerre électronique / capteurs GE), un avion peut identifier dans son environnement ceux qui observent et cherchent potentiellement à le détecter.

\subsection{Chaîne d'information (intégration du capteur)}
Le champs EM  (porteur des impulsions RADAR) arrive sur le capteur ESM, le capteur construit en temps réel des descripteurs pour chaque des impulsions RADAR qu'il détecte (PDW : ToA, LI, Freq, Level, DoA). Cette séquence d'information haut niveau est traité : désentrelacement des impulsions (regrouper en plot par émetteur identique sur un horizon temporel très court), pistage (regrouper les plots provenant des mêmes émetteurs (horizon plus long) et prédiction des trajectoires (je ne suis pas sûr que c'est bien ça en guerre électronique)). Finalement, ces pistes avec la description des caractéristiques du radar associé sont utilisé pour identifier le radar, et déterminer le niveau de menace qu'il représent. Ces dernières informations sont ensuite utilisés pour l'objectif long terme de maintien en vie, que nous ne décrivons pas ici. 
\subsection{Traitement du capteur}
Le champs EM arrive sur les antennes qui créent un signal électrique analogique, ce signal est numérisé et une analyse spectrale sur des fenêtres glissantes permets de suivre les fréquences remarquables à travers le temps. Ce suivi est réalisé avec un principe de mémoire : sur une fenêtre, les fréquences observées sont comparés aux fréquences suivies, si il y a corrélation, la mémoire suivant la fréquence avec corrélation est mise à jour (max une fois par fenêtre) dans le cas contraire, une nouvelle mémoire comment à suivre la fréquence sans corrélation. Lorsqu'une mémoire n'est plus mise-à-jour, il est considéré que l'impulsion à l'origine de la fréquence suivie n'est plus présente et les informations de la mémoire sont agrégés dans un description d'impulsion (PDW) qui est émis en direct.


Ce text était une explication rapide du fonctionnement mais c'est vraiment plus complexe en vrai. Voici un essaie d'explication : 

Le champs EM arrive sur les antennes qui créent un signal électrique analogique. Le signal est numérisé en temps réel. L'objectif ensuite est de déterminer les fréquences présentes dans le spectre de chaque fenêtre temporelle, or pour des raisons de contrainte matériel, l'échantillonnage est réalisé en sous-Nysquist, ce induit du repliement de spectre. Pour retrouver les vrais fréquences, les méthodes de type multi-taux multi-canaux sous-Nyquist (coprime, multi-cosets, Xampling) sont employés (très dominante en RADAR). En pratique (corrige l'explication si faux ou flou), le signal est numérisé plusieurs fois canaux, à différentes fréquences d'échantillonnage. Sur chaque canaux, une analyse spectrale est réalisé pour trouver les fréquences évidentes et en comparaison des fréquences trouvées sur les différents canaux permet de remonter aux vrais fréquences (c'est très mal expliqué ici) sur la fenêtre temporelle considérée. En pratique, sur chaque canal, la fréquence replié d'une impulsion peut être masquées par une autre impulsion de fréquence repliée proche et d'amplitude supérieur, par la saturation du capteur d'une potentielle autre impulsion dont l'amplitude est trop élevée, par les harmoniques d'autres impulsions, par un phénomène d'intermodulation. De plus, sur chaque canal, seul un nombre fixé de fréquence peut être transmis pour la comparaison (raison matériel, compromis de performance), ce qui limite encore le nombre de fréquence effectivement (on conserve les fréquences avec l'amplitude la plus grande). Après comparaison, on se retrouve avec un nombre limité de fréquence (et amplitude) non-ambiguë sur la fenêtre temporelle. Ces fréquences sont transmises au système de suivi temporelle est comparée à des fréquences déjà en cours de suivis par des mémoires. S'il y a corrélation entre une nouvelle fréquence transmise et une mémoire, la mémoire est mise-à-jour et ne peut plus être associé à une fréquence. S'il n'y a pas de corrélation, une nouvelle mémoire commence à suivre la nouvelle fréquence à partir de la date présente, si une mémoire est disponible (elles sont limités pour cause de contrainte matériel). Finalement, c'est lorsqu'une mémoire n'est plus mise-à-jour depuis trop longtemps que ces informations sont regroupées dans un PDW qui est émis et caractérise une impulsion. Cette mémoire est alors de nouveau disponible pour suivre d'autres fréquences.

[Dans ce text, je ne suis pas satisfait avec le niveau de détail de mes explications, je ne sais pas si elles apportent vraiment quelque chose à être aussi précise, je me demande si cité un principe (comme multi-taux multi-canaux sous-Nyquist) et en donner les conséquences sur l'altération des informations n'est pas suffisant. Dans ce cas, il faut trouver un principe qui identifie le mécanisme de suivi.]

On note que le traitement construit une séquence de PDW qui n'ont pas d'assurance d'être la liste des impulsions émises par chaque radar, mais c'est bien ça que le capteur tente de reconstruire. Il se trouve que les PDW produites par le capteur correspondent à la reconstruction des "impulsions", selon une considération experte mais humaine et "rationnelle" de la chose, à partir des traces qui ont pu être identifié du passage de des impulsions émises par le radar dans le signal EM (j'essaie ici de faire passer une compréhension de la nature même des informations reconstituées par le capteur, et de leur complexité en terme d'interprétation, à l'aide). Ainsi, on retrouvera très souvent des PDW résultants de la fusion d'impulsions radar, de fréquence proche et se succédent, ou résultant de la coupure d'une impulsion si il y avait saturation des mémoires ou des antennes, coupure qui peut intervenir au début, au milieu (faisant 2 PDW) ou à la fin d'une impulsion, et une impulsion peut bien entendu ne pas être du tout captée.

\section{Description de l'environnement numérique}
\subsection{Pourquoi un environnement numérique}
Pour l'IVVQ des algorithmes de désentrelacement, pistage et identification, il faut des données d'entrée de ces algorithmes (flux de PDW en sortie de capteur ESM) et les données de sortie (situation tactique). La connaissance exacte de la situation tactique empêche à ce que les données PDW soient interceptés lors d'un vol classique d'un avion. Encore pire, il faut que l'environnement soit parfaitement contrôlé pour éviter des sources dont on ne connaît pas parfaitement le comportement. L'obtention de données réelles contraint l'essaie en vol. La tâche de récolter des données réelles est compliqué, car elle nécessiterait la mise à disposition des moyens très conséquents (faire voler d’autres avions, avoir des radars de surveillance au sol). De plus, ces données ne seraient pas d’une grande diversité de par la limite des moyens disponibles. Dans ce cadre, un simulateur d’impulsions, permettant d’obtenir en fonction d'un scénario prédéfini les impulsions que les capteurs ESM auraient intercepté, est la solution idéale. Ces données peuvent alors être simulées en grande quantité, permettant d’analyser en détail la réaction de notre chaîne d’algorithmes à des scénarios aussi complexes que désirés, et donc de retravailler cette chaîne aux besoins.
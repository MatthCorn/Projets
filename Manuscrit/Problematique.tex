\chapter{Problématique (plan)}

Le chapitre sur la problématique contiendra les éléments suivants:
\begin{itemize}
\item Description du fonctionnement du capteur ESM dans l'environnement (dans la limite de ce qu'on peut dire), intégré dans la chaine algorithmique de traitement de l'information.
\item Description du fonctionnement de l'environnement numérique, avec l'explication des modélisations de chaque traitement.
\item Spécification du goulot d'étranglement et commentaire sur les données I/O.
\item Commentaire sur le complexité du problème pour l'apprentissage automatique : double problématique génération et traitement de séquence.
\item \textcolor{red}{Penser à ajouter des références de traitement avec ce principe de mesureur (brevet ?) pour montrer qu'on ne révèle pas des secrets}
\end{itemize}

\chapter{Problématique}
\section{Introduction}
\textcolor{red}{
Comme abordé dans le chapitre introductif, notre mission est d'accélérer la simulation d'un environnement numérique modélisant l'interception d'impulsion RADAR par des capteurs ESM. Ce chapitre va nous permettre de revenir sur cette problématique. Nous commencerons par l'explication du fonctionnement du capteur ESM et son intégration dans le chaine de traitement de l'information. Nous verrons ensuite pourquoi il est nécessaire de disposer d'un environnement numérique modélisant le fonctionnement du capteur et nous reviendrons sur cette modélisation. APrès nous identifierons le goulot d'étranglement et précierons alors la nature exacte du problème d'accélération que nous aurons à résoudre. Nous porterons une attention particulière à la lecture de se problème sous le spectre de l'apprentissage automatique en notant que l'aspect est à la fois traitement de séquence et génération. Finalement, nous exposerons les références des problèmes où ce type de traitement apparaît, des brevets existants, etc
}
\section{Description de l'intégration du capteur ESM et son fonctionnement}

\subsection{Contexte opérationnel : La maîtrise du spectre électromagnétique} 
Dans le cadre d'un scénario de guerre électronique, la survie de l'aéronef dépend de sa capacité à percevoir et comprendre son environnement électromagnétique. L'appareil évolue dans un espace abondant d'émissions provenant de RADAR adverses ou civils, au sol ou aéroportés, cherchant eux-mêmes à détecter leur cible. Les caractéristiques techniques de ces signaux, telles que la fréquence, la largeur d'impulsion ou la période de répétition, constituent une signature unique permettant d'identifier l'émetteur et d'en déduire ses intentions tactiques (veille, poursuite, engagement). C'est la mission des capteurs de Mesures de Soutien Électronique (ESM - Electronic Support Measures) : assurer une écoute passive et discrète du spectre pour détecter, caractériser et localiser ces menaces potentielles, fournissant ainsi les données critiques à la décision stratégique et aux contre-mesures.

\subsection{Chaîne de traitement de l'information} 
L'intégration du capteur s'inscrit dans une architecture de traitement séquentielle visant à transformer un signal physique brut en renseignement tactique exploitable. Le champ électromagnétique incident est initialement capté par le capteur ESM, qui opère la première conversion fondamentale : il détecte les impulsions radar et construit en temps réel des descripteurs numériques, les PDW (Pulse Description Word - Mot de Description d'Impulsion). Chaque PDW synthétise les paramètres mesurés de l'impulsion : Date d'arrivée (ToA), Largeur d'impulsion (LI), Fréquence, Niveau de puissance et Direction d'arrivée (DoA).

Ce flux continu de PDW alimente ensuite les algorithmes de traitement de haut niveau. Une étape de désentrelacement regroupe d'abord les impulsions par émetteur sur un horizon temporel court, isolant les trains d'impulsions cohérents. Ces regroupements élémentaires sont ensuite consolidés par un processus de pistage (tracking) qui suit l'évolution des émetteurs sur le long terme pour en caractériser la cinématique et le mode de fonctionnement. Finalement, ces pistes enrichies sont confrontées à des bases de données de signatures pour identifier formellement le système d'arme associé et évaluer le niveau de menace immédiat.


\subsection{Traitements du capteur} 
Le fonctionnement interne du capteur ESM ne se limite pas à une conversion analogique-numérique transparente ; il constitue une chaîne complexe de traitements physiques et logiques qui conditionne structurellement la qualité des données produites. Les traitements décrits ci-après constituent le socle architectural de la plupart des récepteurs numériques modernes, bien que des variantes d'implémentation, dictées par les contraintes matérielles des systèmes temps réel, puissent exister selon les constructeurs.

Le traitement débute par la conversion du champ électromagnétique incident en signal électrique analogique. L'objectif est ensuite d'extraire, sur des fenêtres temporelles successives, les raies spectrales significatives. La mission de surveillance de bandes passantes instantanées de plusieurs gigahertz (typiquement $2 - 18 GHz$) impose l'usage de bancs de convertisseurs analogique-numérique (au moins $3$) fonctionnant en parallèle à des cadences inférieures à la fréquence de Nyquist (ex: $1 GHz$, $1.2 GHz$, $1.4 GHz $.) \cite{tsui_digital_2004}. Ce choix architectural induit un repliement spectral systématique sur chaque voie d'acquisition, mais permet la détermination de la fréquence réelle grâce à la résolution d'un système de congruences entre les différentes voies repliées, principe connu sous le nom de théorème des restes chinois \cite{vaidyanathan_sparse_2010}, \cite{li_robust_2009}. Cependant, sur chaque canal d'acquisition, des phénomènes de masquage peuvent intervenir, liés à la saturation des convertisseurs, aux harmoniques et\slash ou à l'intermodulation provoquée par des impulsions de haute énergie, ou à la proximité fréquentielle entre signaux repliés. De plus, la résolution d'ambiguïté s'appuie généralement sur une sélection restreinte des $N$ pics spectraux les plus énergétiques par canal, liée aux contraintes matérielles. La conjonction du bruit thermique sur des canaux critiques et de cette sélection limitative peut conduire à l'échec de la levée d'ambiguïté, entraînant la perte d'impulsions sur la fenêtre temporelle considérée.

Une fois les fréquences non-ambiguës identifiées, elles alimentent un système de suivi temporel chargé de la reconstruction des impulsions. Ce traitement alloue dynamiquement des ressources mémoires, qualifiées de "pistes" \cite{mardia_new_1989}, afin de maintenir la continuité des paramètres mesurés dans le temps et de construire l'objet PDW final \cite{adamy_ew_2001}. La limitation matérielle du nombre de ces mémoires, conjuguée à leur logique d'allocation, engendre des artefacts de segmentation spécifiques. Premièrement, une contrainte de latence maximale impose de segmenter artificiellement les impulsions très longues ou continues pour assurer des mises à jour périodiques, générant une série de PDW contigus. Deuxièmement, les échecs de résolution d'ambiguïté peuvent provoquer une rupture de suivi prématurée. Si le masquage est transitoire, la piste reprend après une interruption, scindant l'impulsion en plusieurs entités. Si le masquage persiste jusqu'à la fin de l'émission, le suivi s'arrête définitivement, tronquant la fin du signal. Troisièmement, la saturation des ressources mémoires en environnement dense impacte directement la détection : si aucune piste n'est disponible à l'apparition du signal, il sera ignoré sur l'instant. Cela conduit soit à une acquisition tardive dès la libération d'une ressource, amputant alors le début du signal, soit à une perte totale de l'impulsion si aucune ressource ne se libère à temps. À l'inverse de ces phénomènes de fragmentation ou de perte, la résolution temporelle finie des bancs de filtres peut conduire à l'amalgame de deux impulsions brèves et rapprochées en un seul descripteur.

Ainsi, la séquence de PDW produite ne doit pas être considérée comme une simple mesure dégradée de la réalité, mais comme une reconstruction interprétée, portant intrinsèquement la signature des limitations fréquentielles et des heuristiques de gestion de ressources du capteur.

\section{Description de l'environnement numérique}
\subsection{Pourquoi un environnement numérique}
Pour l'IVVQ des algorithmes de désentrelacement, pistage et identification, il faut des données d'entrée de ces algorithmes (flux de PDW en sortie de capteur ESM) et les données de sortie (situation tactique). La connaissance exacte de la situation tactique empêche à ce que les données PDW soient interceptés lors d'un vol classique d'un avion. Encore pire, il faut que l'environnement soit parfaitement contrôlé pour éviter des sources dont on ne connaît pas parfaitement le comportement. L'obtention de données réelles contraint l'essaie en vol. La tâche de récolter des données réelles est compliqué, car elle nécessiterait la mise à disposition des moyens très conséquents (faire voler d’autres avions, avoir des radars de surveillance au sol). De plus, ces données ne seraient pas d’une grande diversité de par la limite des moyens disponibles. Dans ce cadre, un simulateur d’impulsions, permettant d’obtenir en fonction d'un scénario prédéfini les impulsions que les capteurs ESM auraient intercepté, est la solution idéale. Ces données peuvent alors être simulées en grande quantité, permettant d’analyser en détail la réaction de notre chaîne d’algorithmes à des scénarios aussi complexes que désirés, et donc de retravailler cette chaîne aux besoins.

\subsection{Fonctionnement de l'environnement}
L'environnement numérique est scindé en 4 parties. Il se base sur un modèle comportementale qui a pour objectif de modéliser l'effet des différents traitements sur le flux d'impulsions afin de déterminer les PDW que le capteur aurait effectivement publié. 

La première partie de l'environnement numérique est un bloc qui permet de lire des fichiers de description de trajectoire (de notre porteur et des radars présents) et les fichiers de description des séquences d'émission de chaque radar (un par radar), afin de modéliser l'émission des impulsions. Ces impulsions émises sont décrites sous la forme de PDW, avec les composantes physiques associées, et constituent une liste dans laquelle elles sont triées par ToA. 

Le deuxième bloc modélise l'effet des antennes. Son rôle premièrement de déterminer l'amplitude des impulsions émises à proximité des antennes de notre porteur. Le calcule est réalisé en fonction la position relative du radar émetteur par rapport à notre porteur, de la direction dans laquelle l'impulsion est émise, et de ses caractéristiques physiques. Et deuxièmement son rôle est de déterminer l'amplitude de l'impact de l'impulsion incidente sur le signal électrique en sortie d'antenne. Ce calcul est effectué à partir du diagramme de Bodes de nos antennes ainsi qu'à partir de l'angle d'incidence de l'impulsion sur nos antennes. La sortie de ce bloc correspond alors aux mêmes PDW qu'en entrée, dont l'amplitude a été modifiée.

Les deux derniers blocs ont pour rôle de modéliser le processus de détection et de caractérisation des impulsions. Ils se placent donc dans une dimension temporelles et fonctionnent parallèlement et en tandem. Cette brique temporelle est appelé le palier, il représente un intervalle temporelle sur lequel les impulsions présentes restent les mêmes. Ainsi, on change de palier si une nouvelle impulsion commence ou si une impulsion en cours se termine. De ce fait, on pourra considérer que les fréquences non-ambiguës qui auraient été détecté par le traitement du capteur auraient été identique sur n'importe quelle fenêtre temporelle incluse dans ce palier.

Pour modéliser les effets de l'analyse fréquentielle et de détection, la premier bloc se place donc à l'échelle d'un palier. Il détermine sur chaque canal comment les impulsions se replieraient et en fonction de leurs niveaux relatives, il détermine pour chaque impulsion si elle est visible ou non. En pratique, on détermine ça en comparant les fréquences et les niveaux des différents impulsions entre elles et à des seuils qui caractérisent le matériel représenté. Finalement, les impulsions détectées (au sens de on a réussis à lever l'ambiguïté sur l'information qui provenait de cette impulsion) sont celles étant visible sur au moins 3 canaux.
\chapter{AVERTISSEMENT}

La composition de la page de couverture doit être respectée pour la diffusion de la thèse sur \url{www.theses.fr} et pour le dépôt légal de la thèse, qui est obligatoire pour l’obtention du diplôme (\href{https://www.legifrance.gouv.fr/affichTexte.do?cidTexte=JORFTEXT000032587086&dateTexte=20160902}{cf. articles 24 et 25 de l’arrêté du 25 mai 2016 fixant le cadre national de la formation et les modalités conduisant à la délivrance du diplôme national de doctorat}).\\ \par
Les consignes et les recommandations ci-après ont pour objet d’assurer une \textbf{homogénéité graphique} pour toutes les thèses soutenues à l’université Paris-Saclay et de les rendre \textbf{immédiatement reconnaissables}.\\ \par
Elles ont également pour objet de donner un cadre de référence permettant d’éviter qu’un lecteur futur puisse avoir des \textbf{doutes sur la conformité de la thèse ou du jury}. L’université reçoit régulièrement des demandes d’informations, au sujet de thèses, pour lesquelles il y a des questionnements sur la conformité du jury ou bien des incohérences entre les informations qui figurent sur la couverture de la thèse, d’une part, et les méta-données de la thèse visibles sur www.theses.fr, d’autre part.\\ \par
Il est rappelé que ces consignes et recommandations ne s’appliquent que pour le dépôt légal de la thèse et sa diffusion via le portail \url{www.theses.fr}. \textbf{Ce canal de diffusion n’est pas exclusif}. D’autres formats de page de couverture peuvent être librement utilisés par les auteurs sur d’autres canaux de diffusion (par exemple : pour afficher le nom et le logo d’une organisation qui aurait co-financé la thèse et pour la diffusion au sein de cette organisation), à condition que les informations requises pour la citation complète de la thèse de doctorat figurent. C’est-à-dire : au minimum : nom et prénom de l’auteur, titre de la thèse, date, lieu et établissement de soutenance (université Paris-Saclay et le cas échéant un établissement partenaire en cas de cotutelle internationale de thèse), ainsi que le logo de l’université Paris-Saclay et le cas échéant d’une université étrangère partenaire en cas de cotutelle internationale de thèse.
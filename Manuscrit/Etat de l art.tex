\chapter{État de l'art : PLAN (à supprimer après rédaction)}

Le chapitre sur l'état de l'art se découpe en 4 parties.

\section{Introduction}

Cette section aborde les aspects suivants :
\begin{itemize}
	\item Environnements numériques
	\item Injection 1 : génération et modélisation de l'environnement
	\item Injection 2 : simulation de phénomènes physiques
	\item Injection 3 : adaptation et interaction
\end{itemize}

\section{IA générative}

Cette section présente le domaine de l'IA générative. Notre problème peut y être naïvement associé mais en réalité quasiment aucune des méthodes ne sera applicable. Les aspects présentés sont:
\begin{itemize}
	\item Le concept d'IA générative. En notant que n'importe quelle fonction génère une sortie à partir d'une entrée et que la dérive de tout appeler IA générative est tentante.
	\item Les VAE et spécialement VAE conditionnels
	\item Les GAN et spécialement GAN conditionnels
	\item Les modèles de diffusion et spécialement ceux conditionnels
	\item Les modèles de langage et GPT
\end{itemize}

\section{Méthodes pour le traitement de séquence}
Cette section présente les architectures connues pour leurs capacités à traiter des séquences, de leurs formes les plus simples aux formes les plus complexes. Par ordre d'apparition:
\begin{itemize}
	\item Le concept de séquence: notion de proximité dans un ensemble. Série temporelle, image, texte.
	\item Réseau de convolution:
	\begin{itemize}
		\item Histoire de son apparition: dans l'image
		\item Comment la convolution interagit avec la séquence
		\item La convolution dans l'image (vue comme une séquence)
		\item La convolution dans le texte
		\item La convolution ailleurs
	\end{itemize}
	\item Réseau de neurones récurrents:
	\begin{itemize}
		\item Histoire de son apparition
		\item Comment un RNN interagit avec la séquence
		\item Variante SSM
		\item RNN dans le texte
		\item RNN dans les systèmes temporels (chaine de Markov)
	\end{itemize}
	\item Transformer:
	\begin{itemize}
		\item Histoire de son apparition: dans le langage
		\item Comment le Transformer interagit avec la séquence
		\item Transformer dans le texte (traduction, GPT, ...)
		\item Transformer dans les systèmes temporels (chaine de Markov)
		\item Transformer dans l'image
		\item Transformer ailleurs (généralisation)
	\end{itemize}
\end{itemize}

\section{Les améliorations}
Cette section met en avant les difficultés liées à l'apprentissage automatique, entre complexité calculatoire, mémorielle et instabilité en entrainement. À cette occasion, nous montrons les propositions existantes visant à résoudre ces problèmes. Par ordre d'apparition:
\begin{itemize}
	\item Compréhension des architectures: Mechanistic Interpretability
	\item Présentation des soucis de performances 
	\item Présentation des solutions aux soucis de performances
	\begin{itemize}
		\item Positional Encoding
		\item Certains mécanismes d'attention
		\item Pre-Training
		\item Embedding et tokenization
	\end{itemize}
	\item Présentation des soucis de stabilité
	\item Présentation des solutions aux soucis de stabilité:
	\begin{itemize}
		\item Layer-norm
		\item Initialisation
		\item Structure (hyper-paramètre de manière générale)
	\end{itemize}
	\item Présentation des soucis d'efficacité et leurs solutions
	\begin{itemize}
		\item Complexité mémoire et calcul: mécanisme d'attention
		\item Vitesse d'entrainement: MAMBA
	\end{itemize}
\end{itemize}

\chapter{État de l'art}

\section{Introduction}

Le concept de jumeau numérique, popularisé et formalisé dès le début des années 2000 par les travaux de Michael Grieves dans le domaine de la manufacturing [\cite{grieves_digital_2015}], puis théorisé comme un pilier des systèmes cyber-physiques (CPS) par des auteurs comme Negri et al. [\cite{negri_review_2017}], a connu une adoption rapide et variée à travers l'industrie.

Si le terme de "jumeau numérique" s'est imposé dans le paysage technologique, sa définition précise fait l'objet d'un débat animé entre une vision idéale et une approche pragmatique. D'un côté, les puristes, s'appuyant sur les travaux fondateurs de la NASA et de Grieves [\cite{grieves_digital_2015}], défendent l'idée qu'un véritable jumeau numérique se caractérise par un couplage bidirectionnel et dynamique avec son homologue physique. Dans cette perspective exigeante, le jumeau n'est pas une simple représentation ; il est un système cyber-physique  qui s'enrichit continuellement des données du physique et, en retour, pilote, optimise et prédit son comportement [\cite{negri_review_2017}]. Cette boucle fermée est considérée comme la condition sine qua non pour distinguer le jumeau numérique d'un simple modèle ou d'une simulation. De l'autre, une approche plus pragmatique, largement répandue dans l'industrie, adopte une définition évolutive et par niveaux de maturité. Dans cette vision, une maquette 3D enrichie de données, parfois qualifiée de "digital shadow", peut déjà être labellisée "jumeau numérique". Cette flexibilité sémantique, bien que source de confusion, reflète la réalité des projets industriels où la complexité et le coût d'une intégration parfaite imposent une progression par étapes. Ainsi, une ligne de démarcation essentielle, cependant, fait consensus : l'existence d'un transfert de données automatique du système physique vers son représentant virtuel. Sans ce flux, la représentation demeure une simulation ou un modèle générique, que nous qualifierons ici d'« environnement numérique ».

L'appropriation du terme « jumeau numérique » dans la littérature et l'industrie recouvre une réalité sémantique complexe, oscillant entre une vision idéale et une application plus pragmatique.  Une ligne de démarcation essentielle, cependant, fait consensus : l'existence d'un transfert de données automatique du système physique vers son représentant virtuel. Sans ce flux, la représentation demeure une simulation ou un modèle générique, que nous qualifierons ici d'« environnement numérique ». Prenons l'exemple du Projet Living Heart de Dassault Systèmes [\cite{baillargeon_living_2014}, \cite{noauthor_projet_2025}] : dans sa forme fondamentale, il s'agit d'une simulation physiologique de très haute fidélité d'un cœur humain « type », un outil remarquable pour la recherche et la formation. Mais il ne devient un jumeau numérique que s'il est individualisé par les données d'imagerie médicale d'un patient spécifique, créant ainsi une copie virtuelle unique de son cœur. De même, les simulateurs de conduite autonome comme CARLA [\cite{dosovitskiy_carla_2017}] sont des environnements numériques essentiels pour l'entraînement des algorithmes d'IA, mais ils simulent un monde routier générique, non couplé à un véhicule physique unique. En revanche, un jumeau numérique de moteur d'ascenseur, alimenté en temps réel par les données de vibration et de température de l'équipement spécifique installé dans un immeuble, incarne la définition minimale du jumeau, souvent appelée « Digital Shadow ». Ainsi, le spectre s'étend du simple environnement numérique (simulation sans données individualisantes) au jumeau numérique mature (couplage bidirectionnel). Pour la suite de cette analyse, nous retiendrons que la caractéristique sin equa non du jumeau numérique est son ancrage dans les données d'une instance physique unique, le distinguant clairement des environnements de simulation plus généraux.

En fonction du domaine d'application, la définition d'un environnement virtuel (Virtual Environment - VE) peut prendre des formes assez différentes. La société de design produit et développement El Passion considère qu'il s'agit d'un \textit{espace numérique qui simule le monde réel ou un environnement complètement nouveau et unique}. En revanche, le Healthcare Simulation Dictionary définit un VE comme \textit{un environnement simulé rendu par ordinateur}, imposant une problématique du rendu visuel propre au domaine de la simulation médicale visant à la formation des praticiens.

Nous proposons la définition suivante:
Un environnement virtuel (VE) désigne une simulation numérique modélisant un ensemble d'entités et de phénomènes, dans le but d'observer, d'analyser ou d'expérimenter des comportements au sein d'un cadre contrôlé.

La notion de VE s'étend donc du monde immersif interactif au banc de test simulé, selon la finalité du dispositif. Dans le contexte du développement algorithmique, un VE permet de reproduire des situations expérimentales, de générer des données synthétiques et de tester des modèles ou des algorithmes sans recourir à des dispositifs physiques.



\chapter{COMPOSITION DU JURY }
La soutenance de la thèse est une évaluation. Les travaux de recherche de doctorat devant être originaux à l’échelle internationale, le Jury est composé sur mesure pour chaque doctorant.e et chaque thèse de doctorat. La composition du Jury est essentielle, le doctorat est délivré par l’université, sous condition du dépôt légal de la thèse, sur avis conforme du Jury. \textbf{Le Jury est garant de la qualité de la thèse}.\\ \par
Pour chacun des membres du Jury, il faut préciser le \textbf{titre}, l’\textbf{affiliation} et la \textbf{fonction dans le jury} sur la page de couverture.

\section{A QUOI SERVENT CES INFORMATIONS ?}
Ces informations doivent permettre, au premier regard, de vérifier la \textbf{conformité de la composition} du Jury :

\begin{itemize}
\renewcommand{\labelitemi}{$\bullet$}
    \item sa \textbf{légitimité} académique pour se prononcer sur l’obtention du plus haut diplôme universitaire, le doctorat (le jury comprend au moins la moitié de professeurs et assimilés et, sauf dérogation, les membres du Jury sont tous eux-mêmes docteurs). 
    \item sa capacité à se prononcer en toute \textbf{indépendance} (au moins la moitié d’externes, à l’établissement de soutenance, à l’école doctorale, à l’équipe d’encadrement, au projet doctoral).
\end{itemize}

\section{LÉGITIMITÉ ACADÉMIQUE}
Les \textbf{titres}  des membres du Jury permettent de vérifier \textbf{qu’au moins la moitié des membres du Jury est professeur} des universités ou assimilé.\\ \par
Les libellés exacts des titres français assimilés aux professeurs des universités (au moins la moitié du Jury) sont disponibles sur \href{https://www.legifrance.gouv.fr/loda/id/LEGITEXT000019860291/}{legifrance}.\\ \par
Le \textbf{président du Jury} est obligatoirement professeur des universités ou assimilé\footnote{\textbf{Arrêté du 15 juin 1992} fixant la liste des corps de fonctionnaires assimilés aux professeurs des universités et aux maîtres de conférences pour la désignation des membres du Conseil national des universités : \url{https://www.legifrance.gouv.fr/affichTexte.do?cidTexte=LEGITEXT000019860291}\\ \par
\textbf{Arrêté du 10 février 2011} relatif à la grille d'équivalence des titres, travaux et fonctions des enseignants-chercheurs mentionnée aux articles 22 et 43 du décret n° 84-431 du 6 juin 1984 fixant les dispositions statutaires communes applicables aux enseignants-chercheurs et portant statut particulier du corps des professeurs des universités et du corps des maîtres de conférences : \url{https://www.galaxie.enseignementsup-recherche.gouv.fr/ensup/pdf/EC_pays_etrangers/Tableau_comparaison_au_26_septembre_2012.pdf}}.\\ \par
Si l’un des rapporteurs n’était pas professeur des universités ou assimilé, il faudrait alors préciser, en plus, qu’il dispose bien de l’HDR (par exemple : Maître de conférences, HDR).

\section{INDÉPENDANCE}
\textbf{Les affiliations} permettent de vérifier que le Jury est bien en \textbf{majorité externe} à l’établissement de soutenance, à l’école doctorale et à l’équipe d’encadrement. 
Pour cela, le nom ou l’acronyme du laboratoire ne suffit pas, en revanche, le nom de l’université ou de l’établissement délivrant le doctorat de rattachement du membre du jury suffit. Il n’est pas utile de préciser certains détails comme l’adresse postale complète ou le pays.\\ \par
Exemple d’affiliation inadaptée car ambiguë : « IJCLab »\\ \par
Lorsque le membre du jury est un chercheur d’un organisme national, fournir le nom de son organisme de rattachement ne suffit pas pour juger de son extériorité (CNRS par exemple). Dans ce cas-là ou dans d’autres cas où il y aurait une incertitude de cette nature, susceptible de susciter des interrogations sur le fait qu’au moins la moitié des membres du Jury est externe, il est alors demandé de préciser, en plus, l’université ou l’établissement où ce chercheur inscrit habituellement ses propres doctorants.\\ \par
\textit{Exemple d’affiliation inadaptée car ambiguë : « CNRS »}\\ \par
Exemple d’affiliation adaptée : \textit{« CNRS, Université de Toulouse »}\\ \par
Lorsqu’il s’agit d’une entreprise ou d’une fondation ou d’une organisation qui n’est pas en lien direct avec un établissement d’enseignement supérieur pour l’inscription de doctorants, il faut alors le préciser.\\ \par
\textit{Par exemple : « Saint Gobain recherche, entreprise »}\\ \par
\textit{Par exemple : « Moveo, Pôle de compétitivité »}
\newpage
\section{FONCTION DANS LE JURY ET ORDRE DE CITATION}
\textbf{La fonction dans le Jury} de chaque membre du Jury doit également être précisée sur la page de couverture.\\ \par
Les fonctions possibles dans un Jury sont : président(e), examinateur ou examinatrice, rapporteur et directeur ou directrice de thèse.
\subsection{Ordre de citation}
Le président du Jury est le premier de la liste. Il est immédiatement suivi des deux rapporteurs dans l’ordre alphabétique, puis des autres examinateurs dans l’ordre alphabétique.\\ \par
Un membre du Jury peut avoir deux fonctions dans le Jury (par ex. rapporteur \& examinateur).
\subsection{Les rapporteurs}
Les rapporteurs participent à l’évaluation de la thèse et figurent donc sur la couverture de thèse, qu’ils soient présents ou non le jour de la soutenance.\\ \par 
\textbf{Si un rapporteur était absent le jour de la soutenance}, il figurerait alors en tant que rapporteur seulement, sinon il figure à la fois en tant que rapporteur \& examinateur.
Lorsqu’un membre du Jury autre qu’un rapporteur, n’a pas pu participer au Jury de soutenance, physiquement ou bien en visioconférence, son nom ne figure pas sur la page de couverture de la thèse. Dans ce cas, il faut veiller à ce que la composition du Jury reste conforme, malgré l’absence du membre du Jury désigné. Cela peut demander de faire passer un membre interne en invité, par exemple.